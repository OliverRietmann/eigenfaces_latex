\section{Gruppenarbeit: Erstellung der Datenbank}
Jedes Programm zur Gesichtserkennung braucht Bilder von Gesichtern, aus denen es \glqq{}lernen\grqq{} kann, wie neue Gesichter zu klassifizieren sind.
Das Lernskript kann mit der bereitgestellten Datenbank bearbeitet werden.
Alternativ kann eine eigene Datenbank erstellt und verwendet werden.
Letzteres lässt sich in als Gruppenarbeit, zum Beispiel mit einer Schulklasse, auslegen.
Die Anleitung dazu befindet sich in Kapitel~\ref{sec:database}.
In dieser Gruppenarbeit noch keine Mathematik vermittelt.
Allerdings bietet sie einen Einstieg in das sonst sehr abstrakte Lernskript, welcher auch der sozialen Komponente des Lernens Rechnung trägt.
Für eine gelungene Gruppenarbeit sind gemäss \cite{Bosch2019} folgende fünf Erfolgsfaktoren entscheidend:
\begin{enumerate}[label=\arabic*.]
	\item \textit{Positive Interdependenz}: Das Ziel kann nur gemeinsam erreicht werden.
	\item \textit{Individuelle Verantwortlichkeit}: Alle müssen ihre Aufgaben
erledigen.
	\item \textit{Förderliche Interaktionen}: Aufgaben müssen soziale Interaktionen erfordern.
	\item \textit{Kooperative Arbeitstechniken}: Führen und sich führen lassen, Konflikte lösen, Kompromisse schliessen.
	\item \textit{Reflexive Prozesse}: Erwerb inhaltlicher und sozialer Kompetenzen.
\end{enumerate}
Bei der gemeinsamen Erstellung einer Datenbank kommen einige dieser Faktoren zu tragen.
Darauf und auch auf einige Probleme wird im Folgenden eingegangen.
\begin{enumerate}[label=\arabic*.]
	\item \textit{Positive Interdependenz}: Je mehr Bilder die Datenbank zu Verfügung hat, desto mächtiger werden die Anwendungen wie zum Beispiel die Bildkompression.
	Dementsprechend werden die Resultate schlechter, wenn nicht jeder seinen Teil zur Datenbank beiträgt. 
	Das Problem ist, dass dies nicht sofort, sondern erst später in der Bearbeitung des Lernskriptes sichtbar wird.
	\item \textit{Individuelle Verantwortlichkeit}: Wie im vorherigen Punkt erklärt, wird das Resultat der Ganzen Klasse schlechter wenn jemand seinen/ihren Beitrag nicht leistet.
	Zudem müssen alle Anweisungen in Kapitel~\ref{sec:database} ganz genau befolgt werden.
	Wenn nur ein einziges Bild nicht das richtige Format hat, wird das Programm nicht laufen und niemand kann das Lernskript weiter bearbeiten.
	\item \textit{Förderliche Interaktionen}: Eine soziale Interaktion ist nur zwingend erforderlich, wenn die Resultate zusammengetragen werden.
	Dabei müssen alle Files gemäss Abbildung~\ref{fig:database} zusammengetragen werden und die richtigen \texttt{.csv} Dateien müssen aufgesetzt werden.
	\item \textit{Kooperative Arbeitstechniken}: Diese sind nur nötig beim Zusammentragen, wie im vorherigen Punkt beschrieben.
	\item \textit{Reflexive Prozesse}: In der Gruppenarbeit wird zwar keine Mathematik unterrichtet, aber das zuschneiden der Bilder kann helfen, ein Bild als eine $M\times N$ \glqq{}Matrix\grqq{} von Pixeln zu verstehen.
\end{enumerate}
In Kapitel~\ref{sec:database} wird nicht darauf eingegangen wie man das Bildbearbeitungsprogramm oder das Tabellen-Kalkulationsprogramm genau verwenden muss.
Der Grund dafür ist einerseits, dass jedes dieser Programme anders funktioniert.
Andererseits sind die Lernenden so gezwungen einander zu helfen.
Sobald jemand herausfindet, wie er z.B. ein Bild zu schwarz-weiss konvertieren kann, sollte er/sie es den anderen zeigen.
Denn die Datenbank kann erst erstellt werden, wenn alle ihren Teil erledigt haben.