\section*{Didaktische Methoden}
Die Vektorgeometrie wir am Gymnasium im Anschauungsraum, also im $\mathbb R^3$, unterrichtet.
Hier lassen sich die wichtigen Konzepte wie Vektoren, Linearkombination, Skalarprodukt, usw. in Bildern darstellen und dadurch leicht erklären.
Während diese Konzepte sich auf natürliche Weise auf höhere Dimensionen verallgemeinern lassen, sind sie dort viel schwieriger darzustellen.
%An die Stelle von Bildern treten komplizierte Formeln mit vielen Indices.
Die Eigengesichter sind ein Weg, bestimmte Vektoren im $\mathbb R^n$ darzustellen.
Vielleicht lässt sich die grundlegende lineare Algebra im $\mathbb R^n$ so auf ansprechende Weise präsentieren.
Die nachfolgenden Kapitel behandeln jeweils eine didaktische Methode.
Dabei wird die Methode erklärt, mit Referenz auf Ergebnisse der Lehr- und Lernforschung.
Zudem wird erklärt, wie diese Methode im Lernskript verwendet wurde, welches den zweiten Teil dieser Arbeit bildet.
