\section{Problem basiertes Lernen}
Hierbei handelt es sich um eine Methode zur kognitiven Aktivierung.
Den Lernenden wird eine Aufgabe zu einem neuen Thema gestellt, ohne dass sie zuvor einen Theorie-Input zu diesem Thema erhalten haben \cite{Loyens2015}.
Dieser folgt erst nach der Bearbeitung dieser Aufgabe.
Dabei soll das Interesse der Lernenden geweckt und ihr Vorwissen aktiviert werden.
Aufgrund des fehlenden Theorie-Inputs mag dieses Vorwissen nicht ausreichend sein um die Aufgabe zu lösen.
In jedem Fall sollen die Lernenden sich aber ihrer Wissenslücken bewusst werden und damit die Notwendigkeit des nachfolgenden Unterrichts besser verstehen \cite{Loyens2015}.
Der Effekt dieses \textit{Problem basierten Lernens} (PBL) wurde in \cite{Loyens2015} untersucht.
Dabei wurden die Newtonschen Bewegungsgesetze mit drei verschiedenen Methoden an drei verschiedene Gruppen unterrichtet.
Diese waren PBL, Selbststudium und ein Lehrervortrag.
Der Inhalt war bei allen Methoden der Selbe.
Aus den nachfolgenden Test geht hervor, dass PBL viel eher zur Überwindung der Messkonzepte führte wie die anderen Methoden.
Zudem war dieser Konzeptwandel insbesondere nach einer Woche beim PBL am effektivsten.

Im Lernskript wurde PBL in Kapitel~\ref{sec:compression} eingesetzt.
Das neue Konzept, welches hier einführt wird, ist die Bildkompression mittels der Eigengesichter.
Die kognitiv aktivierende Aufgabe des PBL Ansatzes ist Aufgabe~\ref{aufg:coef}.
Genauer: Die Grundidee dieser Kompression ist in Abbildung~\ref{fig:coef} enthalten.
Wenn ein Differenzgesicht als Linearkombination der Eigengesichter dargestellt wird, so fallen die Beträge dieser Koeffizienten rasch ab.
Doch bevor erklärt wird, wie nun die Bildkompression mit Eigengesichtern genau funktioniert, sollten die Lernenden Aufgabe~\ref{aufg:coef} bearbeiten.
Dabei vergleichen sie die Linearkombination mittels Eigengesichter mit derjenigen einer anderen Basis des selben Unterraumes.
Der Kontrast dieser beiden Beispiele erlaubt den Lernenden hoffentlich, die speziellen Eigenschaften der Eigengesichter herauszuarbeiten.
Die Frage in Aufgabe~\ref{aufg:coef} ist bewusst offen gestellt, so dass eine Diskussion möglich ist.
Dabei können die Lernenden hoffentlich eine kognitive Aktivierung erfahren, die Ihnen die Auseinandersetzung mit der nachfolgenden Theorie erleichtert.