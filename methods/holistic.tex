\section{Holistic mental model confrontation}
Nachdem ein neues, komplexes Modell unterrichtet wurde bietet es sich an, eine Selbsterklärungsaufgabe zu stellen.
Dabei sollen die Lernenden dieses neue Konzept selber erklären.
Allerdings kann eine weitere Methode der Wissensicherung unter Umständen noch erfolgreicher sein, die \textit{holistic mental model confrontation} (kurz HMMC).
Dabei werden die Lernenden aufgefordert, das (korrekte) Expertenmodell mit einem (fehlerhaften) Laienmodell zu vergleichen.
Dabei sollten im Laienmodell typische Misskonzepte abgebildet sein.
In \cite{Gadgil2012} wurde diese Methode mit einer gewöhnlichen Selbsterklärungsaufgabe verglichen.
Thema war der Blutkreislaufes des Menschen.
Dieser wurde mit einem fehlerhaften und einem exakten Modell dargestellt.
Eine Gruppe von Lernenden musste die zwei Modelle vergleichen, während die andere lediglich das exakte Modell erklären musste.
Es hat sich herausgestellt, dass die Gruppe mit der HMMC das exakte Modell besser verstanden hatte.

Die Auffassung von Bildern als Punkte oder Vektoren im $\mathbb R^n$ sind eine effektive, Modellvorstellung um die Konstruktion der Eigengesichter aus Kapitel~\ref{sec:facespace} zu verstehen.
Dieses Modell ist aber sehr abstrakt, da es sich um Vektoren in $\mathbb R^n$ mit $n\gg 3$ handelt.
Andererseits können diese Vektoren für $n=2$ einfach in ein Koordinatensystem eingetragen werden, auch wenn die Dimensionen dann nicht mehr unserem Anwendungsfall entsprechen.
Genau diese extreme Vereinfachung fassen wir als Laienmodell auf.
In Aufgabe~\ref{aufg:hmmc} soll dieses Laienmodell mit mit dem exakten Modell verglichen werden.