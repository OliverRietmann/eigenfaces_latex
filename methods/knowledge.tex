\section{Vom $\mathbb R^3$ in den $\mathbb R^n$}
Die Vektorgeometrie wird am Gymnasium hauptsächlich im Anschauungsraum $\mathbb R^3$ unterrichtet.
Um die Methode der Eigengesichter zu verstehen, müssen die Konzepte der Vektorgeometrie auf den $\mathbb R^n$ verallgemeinert werden.
Dies birgt insbesondere die zwei folgenden Schwierigkeiten:
\begin{enumerate}[label=\arabic*.]
	\item Es können kaum mehr Bilder gezeichnet werden, um die Konzepte zu veranschaulichen.
	\item Die Komponenten der Vektoren können nicht mehr explizit aufgelistet werden.
\end{enumerate}
Um diese Konzepte effizient zu lernen, muss irgendwie an das Vorwissen im $\mathbb R^3$ angeknüpft werden.
Um dies zu erreichen, wurden im Lernskript die Aufgaben~\ref{aufg:meandiff_simple}, \ref{aufg:scaling_theory} (Kapitel~\ref{sec:eigenfaces}) und~\ref{aufg:projection_3d} (Kapitel~\ref{sec:eigenbasis}) hinzugefügt.
Alle diese Aufgaben behandeln ein Konzept in $\mathbb R^3$, welches anschliessend auf den $\mathbb R^n$ verallgemeinert wird.
Als Beispiel schauen wir uns Aufgabe~\ref{aufg:projection_3d} aus Kapitel~\ref{sec:facespace} genauer an.
In diesem Kapitel geht es darum, einen Vektor $\vec{p}$ auf einen Unterraum zu projizieren, der von einer Familie von orthonormalen Vektoren $\vec{v}_1,\ldots,\vec{v}_n$ aufgespannt wird.
Eine Projektion im $\mathbb R^n$ ist relativ abstrakt.
Als Unterstützung wurde daher zuerst Aufgabe~\ref{aufg:projection_3d} gestellt.
Diese behandelt die analoge Projektion mit Vektoren im $\mathbb R^3$.
Die Komponenten der beteiligten Vektoren sind dabei explizit gegeben und Abbildung~\ref{fig:projection} veranschaulicht die Situation.
Damit können die Lernenden hoffentlich an ihr Vorwissen im $\mathbb R^3$ anknüpfen und den Rest des Kapitels bearbeiten.