\section{Scaffolding}
Selbstständiges Entdecken ist gerade in der Mathematik wünschenswert und wichtig.
Gerade bei komplexen Themen ist das aber schwierig umzusetzen.
In diesem Fall können gewisse Hilfestellungen gegeben werden, die es den Lernenden erlauben, Hindernisse zu überwinden, an denen sie sonst gescheitert wären.
Diese Hilfestellungen bilden sinnbildlich ein \glqq{}Gerüst\grqq{}.
Der Einsatz so eines Gerüstes wird entsprechend als \textit{Scaffolding} bezeichnet \cite{Schnotz2006}.
So eine Hilfestellung ist im Lernskript einerseits durch die Anleitung selbst gegeben, andererseits aber auch durch den zur Verfügung gestellten Python Code.
Dieser nimmt den Lernenden insbesondere folgende Hindernisse aus dem Weg, da die entsprechenden Operationen schon implementiert sind.
\begin{itemize}
	\item Das einlesen der Bilder aus der Datenbank, d.h. die Konvertierung einer Bilddatei in eine Matrix aus Zahlen, wie in Abbildung~\ref{fig:image_to_vector} dargestellt.
	\item Das erstellen von Bildern, d.h. die Konvertierung einer Matrix aus Zahlen in eine Bilddatei.
	\item Die numerisch stabile Berechnung der Eigengesichter mittel Singulärwertzerlegung.
	\item Das einlesen und abspeichern der Metadaten, das heisst der Information, welches Bild der Datenbank zu welcher Person gehört.
\end{itemize}
Ohne diese Hilfestellungen wäre das Thema der Eigengesichter nur schwer zugänglich, denn die Implementierung dieser Operationen kann sehr frustrierend sein.
Andererseits sind sie notwendig um einen graphischen Output zu generieren. 
Zudem haben sie (bis auf die Singulärwertzerlegung) nicht viel mit Mathematik zu tun.
Indem sie als Blackbox zur Verfügung gestellt werden, könne die Lernenden sich auf die wesentliche Mathematik konzentrieren.