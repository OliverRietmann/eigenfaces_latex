\documentclass[12pt,a4paper]{article}
\usepackage[a4paper,total={160mm,250mm}]{geometry}
\usepackage[utf8]{inputenc}
\usepackage[ruled,vlined]{algorithm2e}
\usepackage{amsmath}
\usepackage{amsthm}
\usepackage{amsfonts}
\usepackage{amssymb}
\usepackage{amscd}
\usepackage{array}
\usepackage{caption}
\usepackage{dirtree}
\usepackage{enumitem}
\usepackage{graphicx}
\usepackage{hyperref}
\usepackage{listings}
\usepackage{mathtools}
\usepackage{ngerman}
\usepackage{subcaption}
\usepackage{tcolorbox}
\usepackage{tikz}
\usepackage{xcolor}

\definecolor{shade}{gray}{.5}

\renewcommand{\thesubfigure}{\roman{subfigure}}

\theoremstyle{definition}
\newtheorem{aufgabe}{Aufgabe}

\theoremstyle{definition}
\newtheorem*{losung*}{Lösung}

\theoremstyle{definition}
\newtheorem*{beispiel*}{Beispiel}

\definecolor{mygreen}{rgb}{0,0.6,0}
\definecolor{mygray}{rgb}{0.5,0.5,0.5}
\definecolor{mymauve}{rgb}{0.58,0,0.82}
\definecolor{lightgray}{rgb}{0.9,0.9,0.9}

\lstset{inputpath=../raytracer}
\lstdefinestyle{python}{
	backgroundcolor=\color{lightgray},   % choose the background color; you must add \usepackage{color} or \usepackage{xcolor}; should come as last argument
	basicstyle=\footnotesize\ttfamily,        % the size of the fonts that are used for the code
	breakatwhitespace=false,         % sets if automatic breaks should only happen at whitespace
	breaklines=true,                 % sets automatic line breaking
	captionpos=b,                    % sets the caption-position to bottom
	commentstyle=\color{mygreen},    % comment style
	deletekeywords={...},            % if you want to delete keywords from the given language
	escapeinside={\\[}{\\]},          % if you want to add LaTeX within your code
	extendedchars=true,              % lets you use non-ASCII characters; for 8-bits encodings only, does not work with UTF-8
	firstnumber=1,                   % start line enumeration with line 1000
	frame=single,	                 % adds a frame around the code
	keepspaces=true,                 % keeps spaces in text, useful for keeping indentation of code (possibly needs columns=flexible)
	keywordstyle=\color{blue},       % keyword style
	language=Python,                 % the language of the code
	literate=%
	    {ä}{{\"a}}1%
		{ö}{{\"o}}1%
		{ü}{{\"u}}1%
		{ß}{{\ss}}1%
		{Ä}{{\"A}}1%
		{Ö}{{\"O}}1%
		{Ü}{{\"U}}1,%
	morekeywords={*,...},            % if you want to add more keywords to the set
	numbers=left,                    % where to put the line-numbers; possible values are (none, left, right)
	numbersep=5pt,                   % how far the line-numbers are from the code
	numberstyle=\tiny\color{mygray}, % the style that is used for the line-numbers
	rulecolor=\color{black},         % if not set, the frame-color may be changed on line-breaks within not-black text (e.g. comments (green here))
	showspaces=false,                % show spaces everywhere adding particular underscores; it overrides 'showstringspaces'
	showstringspaces=false,          % underline spaces within strings only
	showtabs=false,                  % show tabs within strings adding particular underscores
	stepnumber=1,                    % the step between two line-numbers. If it's 1, each line will be numbered
	stringstyle=\color{mymauve},     % string literal style
	tabsize=4,	                     % sets default tabsize to 2 spaces
	title=\lstname,                   % show the filename of files included with \lstinputlisting; also try caption instead of title
	rangeprefix=\#---,
	rangesuffix=---,
	includerangemarker=false
}

%\pagestyle{empty}

\title{Eigengesichter}
\author{Oliver Rietmann}
\date{\today}

\begin{document}
\begin{titlepage}
	{\hspace{-1cm}\includegraphics[width=0.3\textwidth]{images/ETHlogo}\par}
	\vspace{1cm}
	\begin{center}
	{\scshape Mentorierte Arbeit in Fachdidaktik Mathematik\par}
	\vspace{1cm}
	{\large\bfseries Eigengesichter\par}
	\vspace{1cm}
	{Oliver Rietmann\par}
	\vfill
	\end{center}
	{\bfseries Inhalt\par}
	{In dieser Arbeit wird die Technik der \textit{Eigengesichter} (engl. \textit{eigenfaces}) erklärt. Hierbei handelt es sich um eine eine rudimentäre Methode zur Gesichtserkennung, welche durch Computer automatisiert werden kann. Unter Gesichtserkennung versteht man klassischerweise das (automatisierte) identifizieren einer Person auf einem Foto. Aber auch damit verwandte Aufgaben werden hier besprochen.
	Diese Arbeit führt die Leser durch eine Anleitung zur Implementierung eines solchen Programms in Python. Der Fokus liegt dabei auf der zugrundeliegenden Mathematik dieses Verfahrens. Die dazu verwendeten Unterrichtsmethoden sollen zudem aus didaktischer Sicht beleuchtet werden.\par}
	\vspace{0.5cm}
	{\bfseries Zielpublikum\par}
	{Fortgeschrittene gymnasiale Mittelschüler mit Schwerpunkt Mathematik und Studenten einer mathematischen/technischer Fachrichtung.\par}
	\vspace{0.5cm}
	{\bfseries Voraussetzungen\par}
	{Vertrautheit mit den Grundbegriffen der linearen Algebra (Vektoren, Matrizen, Basis, Linearkombination, Unterräume von $\mathbb R^n$).
	Zudem werden grundlegende Programmierkenntnisse vorausgesetzt.\par}
	\vspace{0.5cm}
	{\bfseries Form\par}
	{Text mit Aufgaben, Lösungen und Lernzielen. Die Python-Codes befinden sich im Anhang. Alternativ können sie online heruntergeladen werden. (Der Link befindet sich in der Arbeit.)\par}
	\vspace{0.5cm}
	{\bfseries Betreuung\par}
	{Christian Rüede\par}
	\vspace{0.5cm}
	{\bfseries Datum\par}
	{\today\par}
\end{titlepage}
%\maketitle
%\tableofcontents
\clearpage
\section*{Lernskript}
Historisch gesehen war die Methode der Eigengesichter das erste durch Computer automatisierbare Verfahren zur Gesichtserkennung.
Das Fundament dazu wurde 1987 durch Sirovich and Kirby entwickelt \cite{SirovichKirby1987}.
In ihrer Arbeit verwendeten Sie bereits den Begriff \textit{eigenpictures}.
Sie zeigten auf, wie man Fotos von Gesichtern geeignet durch Begriffe der linearen Algebra beschreiben kann.
Damit war der Weg frei um die mächtige Maschinerie der Mathematik, insbesondere der linearen Algebra und der Statistik, auf Fotos anzuwenden.
Dies geschah schon kurz darauf, nämlich 1991, durch Turk und Pentland \cite{Turk1991}.
Sie verwendeten bereits den Begriff \textit{eigenfaces} und beschrieben das Verfahren zur Gesichtserkennung, welches auch wir implementieren werden.
Modernere Methoden wie zum Beispiel \textit{DeepFace} von Facebook funktionieren allerdings viel besser als unsere rudimentäre Methode und sind so gut wie echte Menschen bei der Gesichtserkennung \cite{Taigman2014}.
Ein Vergleich der (heutzutage) besten Methoden findet man zum Beispiel hier \cite{Taskiran2020}.

Alle solchen Programme, auch die modernsten, verwenden eine \textit{Datenbank} von Bildern von Gesichtern, deren Identität bereits bekannt ist.
Man hat also eine gewisse Anzahl von Personen.
Jeder einzelnen Person sind mehrere Bilder zugeordnet, nämlich die Bilder, welche das Gesicht eben dieser Person zeigen.
Man kann sich das als Unterteilung in verschiedene Klassen vorstellen: Zu jeder Person gehört eine Klasse und jede Klasse enthält eine Menge von Bildern.
Dies ist in Abbildung~\ref{fig:database} veranschaulicht.
\begin{figure}[ht]
	\centering
	\begin{tabular}{l m{2cm} m{2cm} m{2cm} m{2cm} c}
		\textbf{Klasse (Person)} & \textbf{Bild 1} & \textbf{Bild 2} & \textbf{Bild 3} & \textbf{Bild 4} & $\cdots$ \\ \hline
		Adam Sandler & \includegraphics[width=0.1\textwidth]{images/intro/class0_0} &
		\includegraphics[width=0.1\textwidth]{images/intro/class0_1} & \includegraphics[width=0.1\textwidth]{images/intro/class0_2} & \includegraphics[width=0.1\textwidth]{images/intro/class0_3} & $\cdots$ \\ \hline
		Emma Watson & \includegraphics[width=0.1\textwidth]{images/intro/class1_0} &
		\includegraphics[width=0.1\textwidth]{images/intro/class1_1} & \includegraphics[width=0.1\textwidth]{images/intro/class1_2} & \includegraphics[width=0.1\textwidth]{images/intro/class1_3} & $\cdots$ \\ \hline
		Natalie Portman & \includegraphics[width=0.1\textwidth]{images/intro/class2_0} & \includegraphics[width=0.1\textwidth]{images/intro/class2_1} & \includegraphics[width=0.1\textwidth]{images/intro/class2_2} & \includegraphics[width=0.1\textwidth]{images/intro/class2_3} & $\cdots$ \\ \hline
		$\qquad\qquad\vdots$ & $\qquad\vdots$ & $\qquad\vdots$ & $\qquad\vdots$ & $\qquad\vdots$ & $\ddots$ \\
	\end{tabular}
	\caption{Visualisierung einer Datenbank von Bildern von Gesichtern.}
	\label{fig:database}
\end{figure}
Aus dieser Datenbank \glqq{}lernt\grqq{} das Programm, neue Bilder zu klassifizieren, also den Personen aus der Datenbank zuzuordnen.
Das Wort \glqq{}neu\grqq{} bedeutet hier, dass dieses Bild nicht notwendigerweise in der Datenbank enthalten ist.
Die Person auf dem Bild muss aber in der Datenbank sein!
Würde die Datenbank in Abbildung~\ref{fig:database} wirklich nur diese drei Personen enthalten, so könnte man zum Beispiel kein Bild von Brad Pitt korrekt klassifizieren, auch wenn eine noch so gute Methode verwendet wird.
Die Datenbank und die Methode der Gesichtserkennung sind unabhängig voneinander.
Das heisst einerseits, aus der selben Datenbank können verschiedene Methoden lernen.
Andererseits kann ein und die selbe Methode verschiedene Datenbanken nutzen.
Wie gut die Gesichtserkennung am Schluss funktioniert hängt nicht nur von der Methode selbst ab, sondern auch von der Datenbank, welche diese verwendet.
Grundsätzlich gilt, dass jede Methode umso besser funktioniert, je mehr Bilder pro Person in der Datenbank gespeichert sind, die sie verwendet.
Mit \glqq{}gut funktionieren\grqq{} ist gemeint, dass neue Bilder mit hoher Wahrscheinlichkeit richtig klassifiziert werden.

Die in Abbildung~\ref{fig:database} gezeigten Bilder stammen aus einer Datenbank von über 10'000 Bildern von über 100 berühmten Persönlichkeiten \cite{Chen14}.
Die Bilder sind alle schwarz-weiss und zeigen lediglich die Gesichter der Personen.
Genau diese Datenbank werden wir auch für alle nachfolgenden Kapitel verwenden \textcolor{green}{(Link zur Datenbank folgt)}.
Allerdings kann auch eine andere Datenbank verwendet werden, sofern sie in das richtige Format gebracht wird \textcolor{green}{(Kapitel dazu folgt)}.

Das Grundgerüst eines Programms zur Gesichtserkennung in Python steht uns schon zur Verfügung.
Wir werden dieses in den folgenden Kapiteln zu einem voll funktionsfähigen Programm erweitern.
Der gesamte Code befindet sich im Anhang \textcolor{green}{(folgt)} und kann unter folgendem Link heruntergeladen werden \textcolor{green}{(Link zum Code folgt)}.
\section{Eigengesichter}
In diesem Kapitel werden wir sehen was die Eigengesichter eigentlich sind.
Zu einer gegebenen Datenbank werden wir diese berechnen und mit unserem Python Code visualisieren.
Allerdings werden wir noch noch keine Gesichtserkennung vornehmen.
\begin{tcolorbox}
	\centerline{\textbf{Lernziele Kapitel 1}}
	\begin{enumerate}[leftmargin=*]
		\item Darstellung eines schwarz-weiss Bildes als Vektor \textit{verstehen}.
		\item Die Begriffe Durchschnittsgesicht, Differenzgesicht und Eigengesicht \textit{verstehen}.
		\item Die Singulärwertzerlegung als Blackbox \textit{anwenden} können.
	\end{enumerate}
\end{tcolorbox}
\subsection{Vom Bild zum Vektor}
Der erste Schritt besteht darin, Bilder als Vektoren aufzufassen.
Das hat zwei Gründe: Erstens können wir diese nur so geeignet in Python darstellen und manipulieren.
Zweitens erlaubt uns das, Bilder in den Kontext der linearen Algebra zu bringen um deren mächtige Methoden anzuwenden.
Als Beispiel betrachten wir ein Bild der Auflösung $N=144$ Pixel (Breite) mal $M=180$ Pixel (Höhe), wie in Abbildung~\ref{fig:image_to_vector}.
Jedem Pixel wird nun eine reelle Zahl zwischen Null und Eins zugeordnet.
Nehmen wir das Pixel an der Stelle $\left(i,j\right)\in\mathbb N^M\times\mathbb N^N$.
Zum Beispiel entspricht $\left(1,N\right)$ dem Pixel in der oberen rechten Ecke des Bildes.
Diesem Pixel wird also eine Zahl $p_{ij}\in\left[0,1\right]$ zugeordnet.
Dabei bedeutet $p_{ij}=0$, dass das Pixel schwarz ist und $p_{ij}=1$, dass es weiss ist.
Die Zahlen dazwischen entsprechen den Graustufen.
Das gibt uns eine $N\times M$-Matrix deren Einträge gerade die $p_{ij}$ sind.
So können wir also ein schwarz-weiss Bild als Matrix auffassen.
Nun schreiben wir die Spalten dieser Matrix in einen Vektor wie in Abbildung~\ref{fig:image_to_vector} gezeigt.
Mit dieser Abbildungsvorschrift können wir jedem schwarz-weiss Bild der Auflösung $N\times M$ auf eindeutige Weise einen Vektor in $\left[0,1\right]^{N\cdot M}$ zuordnen.
Jeder solche Vektor lässt sich auch wieder als Bild darstellen.
Für diesen Schritt ist es egal ob das Bild ein Gesicht zeigt oder etwas anderes.
\begin{figure}[ht]
	\centering
	\begin{tabular}{m{3.5cm} m{1cm} c m{1cm} c}
		\includegraphics[width=0.2\textwidth]{images/vectormatrix/ImageToVector} &
		$\longrightarrow$ &
		$\begin{pmatrix}
			\textcolor{violet}{p_{11}} & \textcolor{orange}{p_{12}} & \cdots & \textcolor{olive}{p_{1N}} \\
			\textcolor{violet}{\vdots} & \textcolor{orange}{\vdots} & \ddots & \textcolor{olive}{\vdots} \\
			\textcolor{violet}{p_{M1}} & \textcolor{orange}{p_{M2}} & \cdots &  \textcolor{olive}{p_{MN}} \\
		\end{pmatrix}$ &
		$\longrightarrow$ &
		$\begin{pmatrix}
			\textcolor{violet}{p_{11}} \\
			\textcolor{violet}{\vdots} \\
			\textcolor{violet}{p_{M1}} \\
			\textcolor{orange}{p_{12}} \\
			\textcolor{orange}{\vdots} \\
			\textcolor{orange}{p_{M2}} \\
			\vdots \\
			\textcolor{olive}{p_{1N}} \\
			\textcolor{olive}{\vdots} \\
			\textcolor{olive}{p_{MN}} \\
		\end{pmatrix}$
	\end{tabular}
	\caption{Ein schwarz-weiss Bild kann als Matrix oder Vektor aufgefasst werden.}
	\label{fig:image_to_vector}
\end{figure}
\pagebreak[4]
\begin{aufgabe}
	Man betrachte das schwarz-weiss Bild, welches durch folgende Matrix beschrieben ist.
	\begin{equation*}
		\begin{pmatrix}
			1 & \frac{1}{4} \\
			\frac{1}{2} & 0 \\
			0 & \frac{3}{4} \\
		\end{pmatrix}
	\end{equation*}
	\begin{enumerate}[label=(\alph*)]
		\item Welche Werte für $N$ und $M$ beschreiben die Auflösung dieses Bildes?
		\item Wie sieht der Vektor aus, der dieses Bild beschreibt?
		\item Welches der folgenden drei Bilder entspricht dieser Matrix?
		
		\definecolor{onefourth}{rgb}{0.25, 0.25, 0.25}
		\definecolor{onehalf}{rgb}{0.5, 0.5, 0.5}
		\definecolor{threefourth}{rgb}{0.75, 0.75, 0.75}
		
		\qquad\qquad
		\begin{tikzpicture}
			\draw[step=1cm,white,very thin] (0,0) grid (2,3);
			\fill[white] (0,0) rectangle (1,1);
			\fill[onefourth] (1,0) rectangle (2,1);
			\fill[onehalf] (0,1) rectangle (1,2);
			\fill[white] (1,1) rectangle (2,2);
			\fill[black] (0,2) rectangle (1,3);
			\fill[threefourth] (1,2) rectangle (2,3);
		\end{tikzpicture}
		\qquad\qquad
		\begin{tikzpicture}
			\draw[step=1cm,white,very thin] (0,0) grid (2,3);
			\fill[black] (0,0) rectangle (1,1);
			\fill[threefourth] (1,0) rectangle (2,1);
			\fill[onehalf] (0,1) rectangle (1,2);
			\fill[black] (1,1) rectangle (2,2);
			\fill[white] (0,2) rectangle (1,3);
			\fill[onefourth] (1,2) rectangle (2,3);
		\end{tikzpicture}
		\qquad\qquad
		\begin{tikzpicture}
			\draw[step=1cm,white,very thin] (0,0) grid (2,3);
			\fill[black] (0,0) rectangle (1,1);
			\fill[onefourth] (1,0) rectangle (2,1);
			\fill[onehalf] (0,1) rectangle (1,2);
			\fill[black] (1,1) rectangle (2,2);
			\fill[white] (0,2) rectangle (1,3);
			\fill[threefourth] (1,2) rectangle (2,3);
		\end{tikzpicture}
	\end{enumerate}
\end{aufgabe}
\begin{losung*}
	Die Lösung der ersten beiden Teilaufgaben kann von Abbildung~\ref{fig:image_to_vector} abgelesen werden.
	Für die letzte Teilaufgabe erinnern wir uns, dass die Zahlen in $\left[0,1\right]$ fliessend den Graustufen von Schwarz (Null) bis Weiss (Eins) entsprechen.
	\begin{enumerate}[label=(\alph*)]
		\item Die Auflösung ist $N=3$ mal $M=2$ Pixel.
		\item Der Vektor ist gegeben durch
		\begin{equation*}
			\begin{pmatrix}
				1 \\
				\frac{1}{2} \\
				0 \\
				\frac{1}{4} \\
				0 \\
				\frac{3}{4} \\
			\end{pmatrix}.
		\end{equation*}
		\item Das mittlere Bild entspricht der Matrix.
	\end{enumerate}
\end{losung*}
In unserem Python Code ist die Funktion, welche eine $N\times M$ Matrix auf diese Weise in einen Vektor der Länge $N\cdot M$ überführt, bereits implementiert.
Sie befindet sich im File \texttt{eigenfaces.py} und heisst \texttt{matrix\_to\_vector}.
Wir betrachten diese nun etwas genauer, um die Manipulation von Matrizen und Vektoren in Python zu lernen.
\begin{lstlisting}[style=python]
import numpy as np

def matrix_to_vector(P, M, N):
	v = np.zeros(M * N)
	for i in range(M):
		for j in range(N):
			v[j + N * i] = P[i, j]
	return v
\end{lstlisting}
Das Argument \texttt{P} ist eine  \texttt{M} mal \texttt{N} Matrix und besteht aus den Einträgen $p_{ij}\in\left[0,1\right]$ wie oben.
Auf die Einträge von Vektoren und Matrizen kann über die eckigen Klammern $[$ und $]$ zugegriffen werden.
Wir brauchen aber auch die Umkehrung dieser Operation.
Das ist der Zweck folgender Übung.
\begin{aufgabe}
	Ergänzen Sie im File \texttt{eigenfaces.py} die Funktion \texttt{vector\_to\_matrix(v, M, N)}.
	Dabei ist \texttt{v} ein Vektor der Länge $\texttt{M}\cdot\texttt{N}$ wie oben.
	Die Funktion soll die zu \texttt{v} gehörende Matrix zurück geben.
	Sie können die ihre Lösung überprüfen indem Sie das Skript \texttt{vector\_to\_matrix\_test.py} laufen lassen.
\end{aufgabe}
\begin{losung*}
	Bei einer richtigen Lösung sollte das Skript \texttt{vector\_to\_matrix\_test.py} das Foto aus Abbildung~\ref{fig:image_to_vector} generieren.
	Die Lösung könnte zum Beispiel so aussehen:
\begin{lstlisting}[style=python]
import numpy as np

def vector_to_matrix(v, M, N):
	P = np.zeros(M, N)
	for i in range(M):
		for j in range(N):
			P[i, j] = v[j + N * i]
	return P
\end{lstlisting}
\end{losung*}
\section{Unterraum der Differenzgesichter} \label{sec:facespace}
\begin{tcolorbox}
	\centerline{\textbf{Lernziele Kapitel~\ref{sec:facespace}}}
	\begin{enumerate}[leftmargin=*,label=\thesection.\arabic*]
		\item Den Durchschnitt einer Familie von Vektoren geometrisch \textit{verstehen}.
		\item Die Translation von Punkten um einen Vektor geometrisch \textit{verstehen}.
		\item Den Begriff \glqq{}Unterraum\grqq{} \textit{erklären} können.
		\item Die Begriffe Durchschnittsgesicht, Differenzgesicht \textit{verstehen}.
	\end{enumerate}
\end{tcolorbox}
Seien nun $M,N\in\mathbb N$ fix.
Wir haben im letzten Kapitel gesehen, wie man schwarz-weiss Bilder der Auflösung $M\times N$ als Vektoren in $\mathbb R^{M\cdot N}$ verstehen kann.
Die Bilder müssen dafür nicht unbedingt ein Gesicht zeigen.
Die Pixel können sogar völlig zufällige Graustufen aufweisen, so dass auf dem Bild nichts sinnvolles zu erkennen ist.
Dies führt uns zu folgender Beobachtung:
Nur die wenigsten Vektoren in $\mathbb R^{M\cdot N}$ entsprechen einem Gesicht.
Wir wollen uns näher mit dieser Beobachtung befassen.

Sei $K\in\mathbb N$ die Anzahl aller Bilder von allen Personen unserer Datenbank.
Jedes Bild soll dabei die Auflösung $M\times N$ haben.
Wir betrachten alle Bilder der Datenbank als Vektoren $\vec b_1,\ldots,\vec b_K\in\mathbb R^{M\cdot N}$.
Diese Darstellung erlaubt uns, das \textit{Durchschnittsgesicht}, wir nennen es $\vec m\in\mathbb R^{M\cdot N}$, zu definieren
\begin{equation*}
	\vec m=\frac{1}{K}\left(\vec b_1+\ldots+\vec b_K\right).
\end{equation*}
Das Durchschnittsgesicht lässt sich wieder als Bild ausgeben.
Aber wie sieht so ein Durchschnittsgesicht aus?
Das werden wir in folgender Übung herausfinden.
\begin{aufgabe}
	Ergänzen Sie im File \texttt{eigenfaces.py} die Funktion \texttt{meanface(b\_list)}.
	Dabei ist \texttt{b\_list} die Liste der Länge $K$ der Vektoren $\vec b_1,\ldots,\vec b_K$.
	Der Rückgabewert soll das Durchschnittsgesicht $\vec m$ sein.
	Sie können die ihre Lösung überprüfen indem Sie das Skript \texttt{meanface\_test.py} laufen lassen.
	\textit{Hinweis:} Die Python Funktionen \texttt{len(...) und sum(...)} können nützlich sein.
\end{aufgabe}
\begin{losung*}
	Hier ist eine mögliche Lösung und das davon mit \texttt{meanface\_test.py} generierte Durchschnittsgesicht.\\[0.5cm]
	\begin{minipage}{0.45\textwidth}
\begin{lstlisting}[style=python]
def meanface(b_list):
	K = len(b_list)
	return sum(b_list) / K
\end{lstlisting}
	\end{minipage}\hfill
	\begin{minipage}{0.3\textwidth}\vspace{-1cm}
		\centering\hfill Durchschnittsgesicht:
	\end{minipage}
	\begin{minipage}{0.2\textwidth}\vspace{-1cm}
		\centering\includegraphics[width=0.6\textwidth]{images/facespace/meanface}
	\end{minipage}
\end{losung*}
Nachdem wir nun das Durchschnittsgesicht gebildet haben, berechnen wir nun die \textit{Differenzgesichter} $\vec a_1,\ldots,\vec a_K$.
Diese sind definiert als
\begin{equation*}
	\vec a_k=\vec b_k-\vec m,\quad k\in\left\{1,\ldots,K\right\}.
\end{equation*}
Die eben eingeführten Begriffe sind links in Abbildung~\ref{fig:meandiff} stark vereinfacht visualisiert.
\begin{figure}[ht]
	\centering
	\begin{minipage}{0.5\textwidth}
		\centering
		\includegraphics[width=\textwidth]{images/facespace/meandiff}
	\end{minipage}\hfill
	\begin{minipage}{0.5\textwidth}
		\centering
		\includegraphics[width=\textwidth]{images/facespace/principal_components}
	\end{minipage}
	\caption{Die Gesichter werden um den Ursprung zentriert indem man das Durchschnittsgesicht subtrahiert (links).
	Die Eigengesichter sind die orthonormalen Vektoren entlang den Hauptachsen (rechts).}
	\label{fig:meandiff}
\end{figure}
\begin{aufgabe}
	Nennen Sie einen Unterschied und eine Gemeinsamkeit der vereinfachten Darstellung links in Abbildung~\ref{fig:meandiff} zu unserer tatsächlichen Situation mit Bildern von Gesichtern.
	Gehen Sie davon aus, dass unsere Bilder eine Auflösung von $M=180$ und $N=144$ haben, wie im letzten Kapitel.
\end{aufgabe}
\begin{losung*}
	Als Vektoren aufgefasst sind die Gesichter Punkte im $\mathbb R^{M\cdot N}$.
	Für $M=180$ und $N=144$ wären das Punkte im $\mathbb R^{25'920}$ und nicht im $\mathbb R^2$ wie in der Abbildung.
	Anders ausgedrückt zeigt die Abbildung den Spezialfall $M\cdot N=2$.
	Das entspricht Bilder die nur aus zwei Pixeln bestehen.
	Andererseits wird in der Abbildung korrekt gezeigt, dass die Komponenten der Gesichts-Vektoren $\vec b_k$ nur Werte zwischen 0 und 1 annehmen.
	Zudem sind die Differenzgesichter richtigerweise genau als Verschiebung der Gesichts-Vektoren um $-\vec m$ dargestellt.
\end{losung*}

Die Eigengesichter werden nun aus den Differenzgesichtern konstruiert.
Wie das genau geht, ist im letzten Kapitel beschrieben.
Hier werden wir nur eine bildliche Konstruktion angeben.
Dazu stelle man sich die Differenzgesichter als eine \glqq{}Wolke\grqq{} von Punkten vor, wie rechts in Abbildung~\ref{fig:meandiff}.
\begin{enumerate}[leftmargin=3cm, label=Schritt \arabic*:]
	\item Entlang einer gewissen Richtung weist diese Wolke die grösste Streuung auf.
	Entlang dieser grössten Streuung wählen wir einen Vektor $\vec u_1$ der Länge 1.
	\item Unter allen Vektoren die orthogonal zu $\vec u_1$ sind, wählen wir wieder einen, der in Richtung der grössten Streuung der Wolke zeigt.
	Diesen nennen wir $\vec u_2$ und er soll wieder Länge 1 haben.
	\item Unter allen Vektoren die orthogonal zu $\vec u_1$ und $\vec u_2$ sind, wählen wir wieder einen, der in Richtung der grössten Streuung der Wolke zeigt.
	Diesen nennen wir $\vec u_3$ und er soll wieder Länge 1 haben.
	\item Analog konstruieren wir $\vec u_4,\vec u_5,\ldots,\vec u_K$.
\end{enumerate}
Die Vektoren $\vec u_1,\ldots,\vec u_K$ heissen \textit{Eigengesichter}.
Sie bilden eine orthonormale Basis des Unterraumes, der von den $K$ Differenzgesichtern aufgespannt wird.
Insbesondere lässt sich jedes der $K$
Wenn man die Komponenten der Eigengesichter geeignet in das Intervall $\left[0,1\right]$ abbildet, lassen sich diese wieder als Bilder darstellen.
%Die Methode der Eigengesichter trifft nun folgende Annahme:
%Die Differenzgesichter sind in guter Approximation in einem niedrig-dimensionalen Unterraum von $\mathbb R^{M\cdot N}$ enthalten.
%Die Dimension dieses Raumes sei $\tilde K$.
%Die obige Annahme bedeutet genauer gesagt, dass $\tilde K\ll K$ und $\tilde K\ll M\cdot N$ (\glqq{}viel kleiner als\grqq{}).
%Wir nennen diesen Unterraum den \textit{Gesichtsraum} (engl. \textit{facespace}).
\section{Eigengesichter als Basis} \label{sec:eigenbasis}
\begin{tcolorbox}
	\centerline{\textbf{Lernziele Kapitel~\ref{sec:eigenbasis}}}
	\begin{enumerate}[leftmargin=*,label=\thesection.\arabic*]
		\item \textit{Verstehen}, was ein Unterraum von $\mathbb R^n$ ist.
		\item Die Eigenschaften \glqq{}orthogonal\grqq{} und \glqq{}orthonormal\grqq{} in $\mathbb R^n$ \textit{verwenden} können.
		\item Die Projektion eines Vektors auf einen Unterraum \textit{ausführen} können, wenn eine orthonormale Basis dieses Unterraumes gegeben ist.
	\end{enumerate}
\end{tcolorbox}
In diesem Unterkapitel wollen wir allgemeine Bilder als Linearkombination der Eigengesichter darstellen.
Man kann zeigen, dass alle Bilder der Datenbank exakt als Linearkombination der Eigengesichter geschrieben werden können.
Neue Bilder können im Allgemeinen mit so einer Linearkombination nur angenähert werden.
Die Beste Annäherung ist gerade die Projektion des neuen Bildes auf den Raum, der durch die Eigengesichter aufgespannt wird.
Was das genau bedeutet, wollen wir zuerst im $\mathbb R^3$ veranschaulichen.
\begin{figure}[ht]
	\centering
	\includegraphics[width=0.5\textwidth]{images/projection}
	\caption{Projektion $\vec a$ von $\vec p$ auf die Ebene durch Null, die von $\vec v_1$ und $\vec v_2$ aufgespannt wird. Der Vektor $\vec n$ steht normal auf die Ebene und es gilt $\vec p=\vec a+\vec n$.}
	\label{fig:projection}
\end{figure}
\begin{aufgabe}
	Betrachten Sie die Vektoren
	\begin{equation*}
		\vec v_1=\frac{1}{\sqrt{3}}\begin{pmatrix}
			1 \\ 1 \\ 1
		\end{pmatrix},\quad
		\vec v_2=\frac{1}{\sqrt{2}}\begin{pmatrix}
			-1 \\ 1 \\  0
		\end{pmatrix},\quad
		\vec p=\begin{pmatrix}
			3 \\ 1 \\  2
		\end{pmatrix}.
	\end{equation*}
	\begin{enumerate}[label=(\alph*)]
		\item Sind die Vektoren $\vec v_1$ und $\vec v_2$ sind orthonormal? Begründen Sie.
		\item Die Vektoren $\vec v_1$ und $\vec v_2$ spannen eine Ebene durch den Nullpunkt auf.
		Berechnen Sie den Punkt $\vec a$, welcher der orthogonalen Projektion von $\vec p$ auf diese Ebene entspricht.
		Dazu können Sie zum Beispiel eine Zerlegung $\vec p=\vec a+\vec n$ wie in Abbildung~\ref{fig:projection} machen.
		\item Dann kann man $\vec a$ als Linearkombination vom $\vec v_1$ und $\vec v_2$ darstellen, also
		\begin{equation*}
			\vec a=c_1\vec v_1+c_2\vec v_2.
		\end{equation*}
		Berechnen Sie die Koeffizienten $c_1$ und $c_2$ in Termen von $\vec v_1,\vec v_2$ und $\vec p$.
	\end{enumerate}
\end{aufgabe}
\begin{losung*}
	\phantom{text}
	\begin{enumerate}[label=(\alph*)]
		\item Die Vektoren sind orthonormal. Wir berechnen das Skalarprodukt
		\begin{equation*}
			\vec v_1\cdot\vec v_2= \frac{1}{\sqrt{3}}\cdot\frac{1}{\sqrt{2}}\cdot\left(1\cdot\left(-1\right)+1\cdot 1+1\cdot 0\right)=0.
		\end{equation*}
		Die beiden Vektoren sind also orthogonal.
		Es bleibt zu zeigen, dass sie beide Länge 1 haben. In der Tat gilt
		\begin{equation*}
			\vec v_1\cdot\vec v_1=\frac{1}{3}\left(1^2+1^2+1^2\right)=1
		\end{equation*}
		und
		\begin{equation*}
			\vec v_2\cdot\vec v_2=\frac{1}{2}\left(\left(-1\right)^2+1^2+0^2\right)=1.
		\end{equation*}
		\item Wir zerlegen $\vec p=\vec a+\vec n$ wie in Abbildung~\ref{fig:projection} in einen Vektor $\vec a$ der in der Ebene liegt und einen Vektor $\vec n$ der orthogonal zur Ebene steht.
		Gesucht ist der Vektor $\vec a$.
		Da er in der Ebene liegt, gibt es eindeutige Koeffizienten $c_1$ und $c_2$, so dass
		\begin{equation*}
			\vec a=c_1\vec v_1+c_2\vec v_2.
		\end{equation*}
		Wir addieren $\vec n$ auf beiden Seiten und erhalten
		\begin{equation*}
			\vec p=c_1\vec v_1+c_2\vec v_2+\vec n.
		\end{equation*}
		Nun bilden wir auf beiden Seiten das Skalarprodukt mit $\vec v_1$, also
		\begin{equation*}
			\vec v_1\cdot\vec p=c_1\underbrace{\vec v_1\cdot\vec v_1}_{=1}+c_2\underbrace{\vec v_1\cdot\vec v_2}_{=0}+\underbrace{\vec v_1\cdot\vec n}_{=0}.
		\end{equation*}
		Hier haben wir verwendet, dass $\vec v_1$ orthogonal ist zu $\vec v_2$ und $\vec n$.
		Wir erhalten
		\begin{equation*}
			c_1=\vec v_1\cdot\vec p=\frac{6}{\sqrt{3}}.
		\end{equation*}
		In dem man stattdessen auf beiden Seiten das Skalarprodukt mit $\vec v_2$ bildet, erhält man analog
		\begin{equation*}
			c_2=\vec v_2\cdot\vec p=-\frac{2}{\sqrt{2}}.
		\end{equation*}
		Damit berechnen wir die Projektion
		\begin{equation*}
			\vec a
			=c_1\vec v_1+c_2\vec v_2
			=2\begin{pmatrix}
				1 \\ 1 \\ 1
			\end{pmatrix}
			-2\begin{pmatrix}
				-1 \\ 1 \\ 0
			\end{pmatrix}
			=\begin{pmatrix}
				4 \\ 0 \\ 2
			\end{pmatrix}.
		\end{equation*}
		\item In der Lösung der vorherigen Teilaufgabe haben wir die Koeffizienten schon berechnet zu
		\begin{equation*}
			c_1=\frac{6}{\sqrt{3}}
			\quad\quad\text{und}\quad\quad
			c_2=-\frac{2}{\sqrt{2}}.
		\end{equation*}
	\end{enumerate}
\end{losung*}

Was wir in der vorherigen Teilaufgabe berechnet haben, funktioniert auch in höheren Dimensionen.
Seien $\vec u,\vec v\in\mathbb R^n$, wobei möglicherweise $n>3$.
Das Skalarprodukt dieser Vektoren ist dann definiert als
\begin{equation*}
	\vec u\cdot\vec v=u_1v_1+\ldots+u_nv_n,
\end{equation*}
wobei $u_1,\ldots,u_n$ die Komponenten von $\vec u$ und $v_1,\ldots,v_n$ die Komponenten von $\vec v$ sind.
Wir sagen, die beiden Vektoren sind orthogonal, wenn $\vec u\cdot\vec v=0$.
Falls sie zusätzlich Länge 1 haben, also $\vec u\cdot\vec u=1$ und $\vec v\cdot\vec v=1$, dann heissen sie orthonormal.
Eine Familie $\vec v_1,\ldots,\vec v_K\in\mathbb R^n$ heisst orthonormal, wenn die Vektoren alle paarweise orthonormal sind.
Sie spannen dann einen $K$-dimensionalen Raum auf.
Wir können wie in der Aufgabe die Projektion eines Vektors $\vec p\in\mathbb n$ auf diesen Raum berechnen, nämlich
\begin{equation*}
	\vec a=c_1\vec v_1+\ldots+c_n\vec v_n,
\end{equation*}
wobei die Koeffizienten gegeben sind durch
\begin{equation*}
	c_i=\vec v_i\cdot p
\end{equation*}
für alle $i\in\left\{1,\ldots,n\right\}$.

Nun betrachten wir wieder die Eigengesichter $\vec u_1,\ldots,\vec u_K$.
Gemäss der Konstruktion aus dem letzten Kapitel sind diese orthonormal.
Wir betrachten zudem die Mona Lisa von Leonardo da Vinci, ein Bild das nicht in unserer Datenbank ist.
Den entsprechenden Vektor nennen wir $\vec p\in\mathbb R^{M\cdot N}$.
Um dieses Bild in den Raum der Differenzgesichter zu verschieben, müssen wir noch das Durchschnittsgesicht $\vec m$ subtrahieren.
Das entstehende Differenzgesicht kann dann näherungsweise als Linearkombination der Basis $\vec a_1,\ldots,\vec a_K$ oder auch der Basis der Eigengesichter $\vec u_1,\ldots,\vec u_K$ dargestellt werden.
Letzteres geht mit einer Projektion wie oben beschrieben.
\begin{aufgabe}
	Sei $\vec p\in\mathbb R^{M\cdot N}$ das Bild der Mona Lisa und $\vec u_1,\ldots,\vec u_K\in\mathbb R^{M\cdot N}$ die Eigengesichter.
	Berechnen Sie die Koeffizienten $c_1,\ldots,c_K$ der Projektion von $\vec p-\vec m$ auf die Eigengesichter.
	Geben Sie die Koeffizienten in Termen von $\vec p, \vec m$ und den Eigengesichtern an.
\end{aufgabe}
\begin{losung*}
	Wie in der letzten Aufgabe verwenden wir, das die Eigengesichter orthonormal sind.
	Wir bilden das Skalarprodukt beider Seiten der obigen Gleichung mit $u_1$, also
	\begin{equation*}
		\vec u_1\cdot\left(\vec p-\vec m\right)=c_1\underbrace{\vec u_1\cdot\vec u_1}_{=1}+c_2\underbrace{\vec u_1\cdot\vec u_2}_{=0}+\ldots+c_K\underbrace{\vec u_1\cdot\vec u_K}_{=0}.
	\end{equation*}
	Wenn wir das auch noch mit $\vec u_2,\ldots,\vec u_K$ machen erhalten wir
	\begin{equation*}
		\vec u_k\cdot\left(\vec p-\vec m\right)=c_k,
	\end{equation*}
	für alle $k\in\{1,\ldots,K\}$.
\end{losung*}

Hat man die Koeffizienten der Linearkombination berechnet, so kann die Projektion der Mona Lisa wieder als Linearkombination der Eigengesichter darstellen
\begin{equation*}
	\vec m+c_1\vec u_1+c_2\vec u_2+\ldots+c_Ku_K.
\end{equation*}
\begin{aufgabe} \label{aufg:compute_coefficients}
	Seien \texttt{p} und \texttt{m} Vektoren der Länge $M\cdot N$, wobei \texttt{p} ein Gesicht zeigt (z.B. das der Mona Lisa) und \texttt{m} das Durchschnittsgesicht ist.
	Sei zudem \texttt{u\_list} die Liste der Eigengesichter.
	Ergänzen Sie die Funktion \texttt{compute\_coefficients(p, m, u\_list)}, welche die Liste der Koeffizienten der Linearkombination aus Abbildung~\ref{fig:eigen_basis} zurück gibt.
	Um Ihre Lösung zu testen können Sie das Python Skript \texttt{basis\_expansion.py} laufen lassen, welches die Projektion der Mona Lisa mit den von Ihnen berechneten Koeffizienten darstellt.
	\textit{Hinweis:} Mit \texttt{np.dot(v,w)} lässt sich das Skalarprodukt zweier Vektoren \texttt{v} und \texttt{w} berechnen.
\end{aufgabe}
\begin{losung*}
	Die Lösung könnte zum Beispiel so aussehen.\\[0.5cm]
	\begin{minipage}{0.65\textwidth}
\begin{lstlisting}[style=python]
import numpy as np

def compute_coefficients(p, m, u_list):
	K = len(u_list)
	c_list = np.empty((K,))
	for c, u in zip(c_list, u_list):
		c = np.dot(u, p - m)
	return c_list
\end{lstlisting}
	\end{minipage}\hfill
	\begin{minipage}{0.35\textwidth}\vspace{-1cm}
		\centering Rekonstruktion:\\[0.5cm]
		\includegraphics[width=0.4\textwidth]{images/eigenfaces/mona_lisa_original}
	\end{minipage}
\end{losung*}

Anscheinend sieht die Projektion fast gleich aus wie das Bild selber.
Etwas informell ist das in in Abbildung~\ref{fig:eigen_basis} dargestellt.
\begin{figure}[ht]
	\centering
	\begin{tabular}{m{1.8cm} c c c c c m{2cm} c c c m{2cm} c c}
		\includegraphics[width=0.1\textwidth]{images/eigenfaces/mona_lisa_original} &
		$\approx$ & $\vec m$ & $+$ & $c_1$ & $\cdot$ & \includegraphics[width=0.1\textwidth]{images/eigenfaces/eigenface00}
		& $+$ & $c_2$ & $\cdot$ & \includegraphics[width=0.1\textwidth]{images/eigenfaces/eigenface01} & $+$ & $\cdots$
	\end{tabular}
	\caption{Projektion der Mona Lisa auf die Eigengesichter.}
	\label{fig:eigen_basis}
\end{figure}
Dieses Phänomen bedeutet, dass die Mona Lisa ganz oder in sehr guter Näherung im Raum liegt, er durch die Eigengesichter aufgespannt wird.
In Abbildung~\ref{fig:projection} würde das einem $\vec p$ entsprechen, dass mit der Ebene einen verschwindend kleinen Winkel einschliesst, also fast in der Ebene liegt.
Mit dieser Beobachtung wollen wir und im nächsten Kapitel genauer auseinandersetzen.
\subsection{Berechnung der Eigengesichter}
\textcolor{green}{TODO: Eigengesichter berechnen mit Singulärwertzerlegung}\\

\begin{tcolorbox}
	\centerline{\textbf{Lernzielkontrolle Kapitel 1}}
	\begin{aufgabe}
		Frage zu Lernziel 1 folgt...
	\end{aufgabe}
	\begin{aufgabe}
		Frage zu Lernziel 2 folgt...
	\end{aufgabe}
	\begin{aufgabe}
		Frage zu Lernziel 3 folgt...
	\end{aufgabe}
	\begin{aufgabe}
		Frage zu Lernziel 4 folgt...
	\end{aufgabe}
\end{tcolorbox}

\subsection{Didaktische Methoden}
\begin{itemize}
	\item interleaved practice: Anstatt zuerst die ganze Theorie zu entwickeln und anschliessend zu programmieren, wechseln die Aufgaben ab: Es kommen abwechslungsweise Theorie- und Programmieraufgaben.
	Diese \textit{interleaved practice} verspricht langfristig besseren Lernerfolg verglichen mit der sequenziellen Alternative \cite{Rohrer14}.
	\item Unterteilung der Lernziele nach der revidierten Taxonomie von Bloom \cite{Anderson2001}.
	\item ...
\end{itemize}
\nocite{*}
\bibliographystyle{plain}
\bibliography{references}
\end{document}