\section{Eigengesichter}
In diesem Kapitel werden wir sehen was die Eigengesichter eigentlich sind.
Zu einer gegebenen Datenbank werden wir diese berechnen und mit unserem Python Code visualisieren.
Allerdings werden wir noch noch keine Gesichtserkennung vornehmen.
\begin{tcolorbox}
	\centerline{\textbf{Lernziele Kapitel 1}}
	\begin{enumerate}[leftmargin=*]
		\item Darstellung eines schwarz-weiss Bildes als Vektor \textit{verstehen}.
		\item Die Begriffe Durchschnittsgesicht, Differenzgesicht und Eigengesicht \textit{verstehen}.
		\item Die grundlegenden Matrix und Vektor Operation in Python \textit{anwenden} können (Addition, Multiplikation, Einträge auslesen oder verändern).
		\item Die Singulärwertzerlegung als Blackbox \textit{anwenden} können.
	\end{enumerate}
\end{tcolorbox}