\section{Berechnung der Eigengesichter: Singulärwertzerlegung}
Wir betrachten die Matrix, deren Spalten aus den Bildern der Datenbank bestehen, also
\begin{equation*}
	A=\left(\vec a_1,\ldots,\vec a_K\right).
\end{equation*}
Wenn wir diese Matrix auf einen Vektor $\vec c\in\mathbb R^K$ anwenden, dann berechnet diese gerade die Linearkombination der Bilder $\vec a_1,\ldots,\vec a_K$ und die Komponenten von $\vec c$ sind gerade die Koeffizienten dieser Linearkombination
\begin{equation*}
	Ac=c_1\vec a_1+\ldots+c_K\vec a_K.
\end{equation*}
Der Vektor $A\vec c$ liegt wieder in $\mathbb R^{M\cdot N}$ und kann als Bild interpretiert werden, falls dessen Einträge in $\left[0,1\right]$ liegen.
In diesem Sinne ist $A\vec c$ eine gewichtete Überlagerung der Bilder/Gesichter der Datenbank und liegt $A\vec c$ im Raum der Differenzgesichter.
\begin{aufgabe}
	Welches Format hat die Matrix $A$, d.h. wie viele Zeilen und Spalten hat sie?
	Geben sie dies durch die Variablen $K,M$ und $N$ an. 
\end{aufgabe}
\begin{losung}
	Sie hat das Format $\left(M\cdot N\right)\times K$.
	Das heisst sie hat $M\cdot N$ Zeilen (eine für jedes Pixel) und $K$ Spalten (eine für jedes Bild in der Datenbank).
\end{losung}
Nun verwenden wir (ohne Herleitung) ein klassisches Resultat aus der linearen Algebra, die \textit{Singulärwertzerlegung}.
Zudem nehmen wir an, dass $K\leq M\cdot N$, also dass die Anzahl der Bilder der Datenbank kleiner oder gleicher der Anzahl Pixel eines Bildes ist.
In diesem Fall besitzt die Matrix $A$ besitzt eine Zerlegung der Form
\begin{equation*}
	A=U\Sigma V^T,
\end{equation*}
wobei $U\in\mathbb R^{\left(M\cdot N\right)\times K},V\in\mathbb R^{\left(M\cdot N\right)\times K}$ und $\Sigma\in\mathbb R^{K\times K}$ Matrizen sind, mit folgenden Eigenschaften:
\begin{itemize}
	\item Die Spalten $u_1,\ldots,u_K\in\mathbb R^{M\cdot N}$ der Matrix $U$ sind orthonormal.
	\item Die Spalten $v_1,\ldots,v_K\in\mathbb R^{M\cdot N}$ der Matrix $V$ sind orthonormal.
	\item Die Matrix $\Sigma$ ist diagonal mit den Einträgen $\sigma_1\geq\ldots\geq\sigma_K\geq 0$.
	Diese Diagonal-Einträge heissen \textit{Singulärwerte} von $A$.
\end{itemize}
Eine Zerlegung dieser Form heisst Singulärwertzerlegung.
Schematisch kann man das auch folgendermassen darstellen
\begin{equation}\label{eq:svd}
A=
\begin{pmatrix}
	\vert & \vert & \cdots & \vert \\
	u_1   & u_2   & \cdots & u_K \\
	\vert & \vert & \cdots & \vert
\end{pmatrix}
\,
\begin{pmatrix}
	\sigma_1 & & & \\
	& \sigma_2 & & \\
	& & \ddots & \\
	& & & \sigma_K
\end{pmatrix}
\,
\begin{pmatrix}
	\text{---} & v_1^T & \text{---} \\
	\text{---} & v_2^T & \text{---} \\
	\vdots & \vdots & \vdots \\
	\text{---} & v_K^T & \text{---}
\end{pmatrix}.
\end{equation}
Wir haben vorher gesehen, dass die Matrix-Vektor Multiplikation $Ac$ eigentlich eine Linearkombination der Spaltenvektoren von $A$ bildet und diese Spalten sind genau die Bilder aus der Datenbank.
Doch über die Singulärwertzerlegung können wir den Vektor $Ac$ auch als Linearkombination der Spaltenvektoren $u_1,\ldots,u_K$ von $U$ interpretieren.
Dazu ersetzen wir im Term $Ac$ die Matrix $A$ durch ihre Singulärwertzerlegung in Gleichung~\ref{eq:svd}.
Dann berechnen wir die Matrix-Vektor Multiplikationen von rechts nach links.
Das heisst wir beginnen mit $V^Tc$, also
\begin{equation*}
	V^Tc=
	\begin{pmatrix}
		\text{---} & v_1^T & \text{---} \\
		\text{---} & v_2^T & \text{---} \\
		\vdots & \vdots & \vdots \\
		\text{---} & v_K^T & \text{---}
	\end{pmatrix}
	\begin{pmatrix}
		c_1 \\
		c_2 \\
		\vdots \\
		c_{M\cdot N}
	\end{pmatrix}=
	\begin{pmatrix}
		v_1^T c \\
		v_2^T c \\
		\vdots \\
		v_K^T c
	\end{pmatrix}.
\end{equation*}
Die Komponenten des Vektors $V^Tc$ sind also die Skalarprodukte der $v_k$ mit $c$.
Folglich gilt
\begin{equation*}
	\Sigma V^Tc=
	\begin{pmatrix}
		\sigma_1 & & & \\
		& \sigma_2 & & \\
		& & \ddots & \\
		& & & \sigma_K
	\end{pmatrix}\,
	\begin{pmatrix}
		v_1^T c \\
		v_2^T c \\
		\vdots \\
		v_K^T c
	\end{pmatrix}=
	\begin{pmatrix}
		\sigma_1 v_1^T c \\
		\sigma_2 v_2^T c \\
		\vdots \\
		\sigma_K v_K^T c
	\end{pmatrix}.
\end{equation*}
Wenn wir den Vektor auf der rechten Seite mit $\tilde c$ bezeichnen, also $\tilde c_k=\sigma_k v_k^T c$, dann erhalten wir schliesslich
\begin{equation*}
	Ac = U\Sigma V^Tc=U\tilde c
	=\tilde c_1 u_1+\ldots+\tilde c_K u_K.
\end{equation*}
Hier sehen wir bereits, dass das $Ac$ eine Linearkombination der Vektoren $u_1,\ldots,u_K$ ist.
Wenn wir für $\tilde c_k$ wieder $\sigma_k v_k^T c$ ersetzen, können wir das noch ausschreiben als
\begin{equation}\label{eq:span_uk}
	Ac =\left(\sigma_1 v_1^Tc\right) u_1+\ldots+\left(\sigma_K v_K^Tc\right) u_K.
\end{equation}
\begin{aufgabe}
	Gleichung~\eqref{eq:span_uk} sagt, dass jede Linearkombination von Bildern aus der Datenbank $a_1,\ldots,a_K$ auch als Linearkombination der Spalten $u_1,\ldots,u_K$ von $U$ geschrieben werden kann.
	Erklären Sie, warum das so ist.
\end{aufgabe}
\begin{losung}
	Wir haben gesehen, dass für jeden Vektor $c\in\mathbb R^K$ die Matrix-Vektor Multiplikation $Ac$ gerade die Linearkombination
	\begin{equation*}
		c_1u_1+\ldots+c_Ku_K
	\end{equation*}
	der Spalten von $A$ bildet.
	Durch Variation von $c$ kann so jede mögliche Linearkombination der Bilder aus der Datenbank generiert werden.
	Doch zu jeder solcher Linearkombination liefert Gleichung~\eqref{eq:span_uk} eine entsprechende Linearkombination der $u_1,\ldots,u_K$ die den gleichen Vektor darstellt.
\end{losung}
\begin{aufgabe}
	Sei $k\in\left\{1,\ldots,K\right\}$ beliebig.
	Schreiben Sie das $k$-te Bild aus der Datenbank, also $a_k$, also Linearkombination der $u_1,\ldots,u_K$.
	Das heisst, geben Sie die Koeffizienten dieser Linearkombination an in Termen von $\sigma_1,\ldots,\sigma_K$ und $v_1,\dots,v_K$.
\end{aufgabe}
\begin{losung}
	Wir wählen in Gleichung~\ref{eq:span_uk} den Vektor $c$ so, dass $Ac$ genau das Gesicht $a_k$ ausliest.
	Dazu müssen alle Komponenten von $c$ Null sein, ausser die $k$-te Komponente ist Eins, also
	\begin{equation*}
		c^T=\left(0\ \ldots\ 0\ \smash{\overbrace{1}^{k\textrm{-te Komponente}}}\ 0\ \ldots\ 0\right).
	\end{equation*}
	Nun gilt $a_k=Ac$.
	Mit dieser Wahl von $c$ liefert Gleichung~\eqref{eq:span_uk} die gewünschte Linearkombination
	\begin{equation*}
		a_k=\left(\sigma_1 v_1^Tc\right) u_1+\ldots+\left(\sigma_K v_K^Tc\right) u_K.
	\end{equation*}
\end{losung}

\begin{tcolorbox}
	\centerline{\textbf{Lernzielkontrolle Kapitel 1}}
	\begin{aufgabe}
		Frage zu Lernziel 1 folgt...
	\end{aufgabe}
	\begin{aufgabe}
		Frage zu Lernziel 2 folgt...
	\end{aufgabe}
	\begin{aufgabe}
		Frage zu Lernziel 3 folgt...
	\end{aufgabe}
	\begin{aufgabe}
		Frage zu Lernziel 4 folgt...
	\end{aufgabe}
\end{tcolorbox}

\subsection{Didaktische Methoden}
\begin{itemize}
	\item interleaved practice: Anstatt zuerst die ganze Theorie zu entwickeln und anschliessend zu programmieren, wechseln die Aufgaben ab: Es kommen abwechslungsweise Theorie- und Programmieraufgaben.
	Diese \textit{interleaved practice} verspricht langfristig besseren Lernerfolg verglichen mit der sequenziellen Alternative \cite{Rohrer14}.
	\item Unterteilung der Lernziele nach der revidierten Taxonomie von Bloom \cite{Anderson2001}.
	\item ...
\end{itemize}