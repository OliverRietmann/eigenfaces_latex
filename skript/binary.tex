\section{Geschlechtserkennung} \label{sec:binary}
Im letzten Kapitel haben wir die Bildkompression mittels Eigengesichter kennengelernt.
Diese eröffnet viele neue Möglichkeiten.
Eine davon ist die \textit{Geschlechtserkennung}.
Damit ist gemeint, dass man zu einem gegebenen Gesicht $\vec p\in\mathbb R^{M\times N}$ entscheidet, ob es einen Mann oder eine Frau zeigt.
Dazu verwenden wir ein bekanntes Konzept aus der Vektorgeometrie, die Ebene, oder genauer gesagt, die Hyperebene.