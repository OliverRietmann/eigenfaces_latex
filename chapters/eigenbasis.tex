\subsection{Eigengesichter als Basis}
Im letzten Unterkapitel haben wir den Raum der Differenzgesichter eingeführt.
In diesem Unterkapitel wollen wir eine geeignete Basis dieses Unterraumes finden.
%Die Bilder $\vec a_1,\ldots,\vec a_K$ aus unserer Datenbank sind im Allgemeinen nicht paarweise orthogonal und nicht auf Länge 1 normiert.
Wir werden im Folgenden annehmen, dass die Anzahl der Bilder $K$ viel kleiner ist als die Anzahl der Pixel $M\cdot N$ der einzelnen Bilder.
Der Einfachheit halber nehmen wir zudem an, dass die Vektoren $\vec a_1,\ldots,\vec a_K$ linear unabhängig sind.
Diese bilden dann die Basis eines $K$-dimensionalen Unterraums von $\mathbb R^{M\cdot N}$.
Nun betrachten wir eine weitere Basis dieses Unterraumes, die \textit{Eigengesichter}.
Die Eigengesichter sind eine orthonormale Basis dieses Raumes, wir bezeichnen diese mit $\vec u_1,\dots,\vec u_K\in\mathbb R^{M\cdot N}$.
Diese Basis ist in vielerlei Hinsicht speziell, wie wir sehen werden.
Wenn wir sie um das Durchschnittsgesicht $\vec m$ verschieben, könne wir die Vektoren wieder als Bilder darstellen.
Das ist in Abbildung~\ref{fig:eigenfaces} gezeigt.
\begin{figure}[ht]
	\centering
	\begin{tabular}{cccccccc}
		\includegraphics[width=0.1\textwidth]{images/eigenfaces/eigenface00} & \includegraphics[width=0.1\textwidth]{images/eigenfaces/eigenface01} &
		\includegraphics[width=0.1\textwidth]{images/eigenfaces/eigenface02} & \includegraphics[width=0.1\textwidth]{images/eigenfaces/eigenface03} &
		\includegraphics[width=0.1\textwidth]{images/eigenfaces/eigenface04} &
		\includegraphics[width=0.1\textwidth]{images/eigenfaces/eigenface05} & \includegraphics[width=0.1\textwidth]{images/eigenfaces/eigenface06} &
		\includegraphics[width=0.1\textwidth]{images/eigenfaces/eigenface07} \\ \includegraphics[width=0.1\textwidth]{images/eigenfaces/eigenface08} &
		\includegraphics[width=0.1\textwidth]{images/eigenfaces/eigenface09} & \includegraphics[width=0.1\textwidth]{images/eigenfaces/eigenface10} &
		\includegraphics[width=0.1\textwidth]{images/eigenfaces/eigenface11} & \includegraphics[width=0.1\textwidth]{images/eigenfaces/eigenface12} &
		\includegraphics[width=0.1\textwidth]{images/eigenfaces/eigenface13} & \includegraphics[width=0.1\textwidth]{images/eigenfaces/eigenface14} &
		\includegraphics[width=0.1\textwidth]{images/eigenfaces/eigenface15} \\
	\end{tabular}
	\caption{Die ersten 16 Eigenfaces wurden wieder als Bild dargestellt.}
	\label{fig:eigenfaces}
\end{figure}
Wir betrachten nun die Mona Lisa von Leonardo da Vinci, ein Bild das nicht in unserer Datenbank ist.
Auch hier können wir durch Subtraktion von $\vec m$ das Differenzgesicht bilden.
Nun ist es möglich, dieses Differenzgesicht näherungsweise als Linearkombination der Basis $a_1,\ldots,a_K$ oder auch der Basis der Eigengesichter $u_1,\ldots,u_K$ zu schreiben.
Letzteres ist in Abbildung~\ref{fig:eigen_basis} dargestellt.
\begin{figure}[ht]
	\centering
	\begin{tabular}{m{1.8cm} c c c c c m{2cm} c c c m{2cm} c c}
		\includegraphics[width=0.1\textwidth]{images/eigenfaces/mona_lisa_original} &
		$-$ & $\vec m$ & $\approx$ & $c_1$ & $\cdot$ & \includegraphics[width=0.1\textwidth]{images/eigenfaces/eigenface00}
		& $+$ & $c_2$ & $\cdot$ & \includegraphics[width=0.1\textwidth]{images/eigenfaces/eigenface01} & $+$ & $\cdots$
	\end{tabular}
	\caption{Differenzgesicht der Mona Lisa als Linearkombination der Eigenfaces.}
	\label{fig:eigen_basis}
\end{figure}
Wir gehen hier nicht näher darauf ein, wie man die Koeffizienten $c_1,\ldots,c_K$ dieser Linearkombination findet (das kommt später).
Wenn mab aber als Basis nicht die Eigengesichter, sondern die Gesichter der Datenbank verwendet hätte, so würden die Koeffizienten anders aussehen, was in Abbildung~\ref{fig:coef} gezeigt ist.
\begin{figure}[ht]
	\centering
	\begin{tabular}{lr}
		\includegraphics[width=0.45\textwidth]{images/eigenfaces/naive_coef} & \includegraphics[width=0.45\textwidth]{images/eigenfaces/eigen_coef} \\
	\end{tabular}
	\caption{Der Absolutbetrag der Koeffizienten der Linearkombination des Differenzgesichts der Mona Lisa. Links wurden als Basis die Gesichter der Datenbank und rechts die Eigengesichter genommen.}
	\label{fig:coef}
\end{figure}
\begin{aufgabe}
	Betrachten Sie die Abbildung~\ref{fig:coef}, welche die Koeffizienten der Linearkombinationen bezüglich der beiden Basen zeigt.
	Beschreiben Sie die Unterschiede der Bilder.
	Welche Rückschlüsse kann man dadurch auf die entsprechenden Basen ziehen?
\end{aufgabe}
\begin{losung*}
	Die Koeffizienten der Eigengesichter fallen schnell ab.
	Folglich leisten die ersten $\approx 1000$ Eigengesichter den grössten Beitrag zur Linearkombination.
	Bei der Basis $a_1,\ldots,a_K$ ist keine Solche Struktur zu erkennen.
	In diesem Sinn sind alle Bilder etwa gleich wichtig um das Bild der Mona Lisa darstellen zu können.
\end{losung*}
Was wir hier in Abbildung~\ref{fig:coef} beobachtet haben ist kein Einzelfall.
Obwohl wir nur das Beispiel der Mona Lisa betrachtet haben, würden andere Gesichter ähnlich verteilte Koeffizienten liefern.
Im nächsten Unterkapitel werden wir die Eigengesichter berechnen.
%\begin{figure}[ht]
%	\centering
%	\begin{tabular}{lr}
%		\includegraphics[width=0.2\textwidth]{images/eigenfaces/mona_lisa_original} & \includegraphics[width=0.2\textwidth]{images/eigenfaces/mona_lisa_2000} \\
%	\end{tabular}
%	\caption{Mona Lisa von Leonardo da Vinci. Links ist das original, rechts ist die Approximation als Linearkombination der ersten 2000 Eigengesichter.}
%	\label{fig:mona_lisa}
%\end{figure}