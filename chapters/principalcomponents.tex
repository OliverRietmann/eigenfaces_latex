\subsection{Berechnung der Eigengesichter}
\textcolor{green}{TODO: Eigengesichter berechnen mit Singulärwertzerlegung}\\

\begin{tcolorbox}
	\centerline{\textbf{Lernzielkontrolle Kapitel 1}}
	\begin{aufgabe}
		Frage zu Lernziel 1 folgt...
	\end{aufgabe}
	\begin{aufgabe}
		Frage zu Lernziel 2 folgt...
	\end{aufgabe}
	\begin{aufgabe}
		Frage zu Lernziel 3 folgt...
	\end{aufgabe}
	\begin{aufgabe}
		Frage zu Lernziel 4 folgt...
	\end{aufgabe}
\end{tcolorbox}

\subsection{Didaktische Methoden}
\begin{itemize}
	\item interleaved practice: Anstatt zuerst die ganze Theorie zu entwickeln und anschliessend zu programmieren, wechseln die Aufgaben ab: Es kommen abwechslungsweise Theorie- und Programmieraufgaben.
	Diese \textit{interleaved practice} verspricht langfristig besseren Lernerfolg verglichen mit der sequenziellen Alternative \cite{Rohrer14}.
	\item Unterteilung der Lernziele nach der revidierten Taxonomie von Bloom \cite{Anderson2001}.
	\item ...
\end{itemize}